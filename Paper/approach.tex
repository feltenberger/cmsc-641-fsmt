\section{Proposed Approach}\label{approach}

\subsection{System Model}
The model and algorithms described in this paper make the following assumptions:

\begin{itemize}
\item Each sensor/node sends an equal amount of data per time unit.
\item Each node can receive an arbitrary amount of data per time unit.
\item Each node can store an arbitrary amount of data without energy penalties. This assumption enables this model to ignore data transmission bottlenecks.
\item The transmission range of each sensor node does not exceed the maximum transmission range $T$.
\item Only nodes within transmission range of each other are connected (i.e. have an edge between each other in the network).
\item A node's energy dissipation per bit transmitted (according to the first order radio model \cite{820485}).
\item Transmitting a bit over distance $d$ requires energy $d^2$.
\item Sensor nodes have a fixed location. Their location may not change during the lifetime of the network.
\item Relay nodes are movable. Their location is flexible during the lifetime of the network.
\item Relay nodes can be added to the original network.
\end{itemize}

\subsection{Problem Formulation}

In the Minimal Fair Energy Consumption with Minimal Additional Resources (MFEC-MAR) problem the goal, given a wireless sensor network represented as a graph $G=(V \cup S,E)$, is to equalize the power consumption rate $PCR$ of all nodes in $G$. That is $PCR(v_1)=PCR(v_2)=PCR(v_3)=...=PCR(v_n)$ where $n=|V \cup S|$ denotes the number of vertices/nodes in $G$. The set of vertices in $G$ is composed of a set of fixed vertices $V$ (i.e. sensor nodes) and a set of moveable vertices $S$ (i.e. relay nodes). The edges $E$ in $G$ are subject to the maximum transmission range $T$ of the nodes. It is furthermore the goal of MFEC-MAR to minimize overall power consumption of the network such that $\sum_{i=1}^{n} PCR(v_i)$ is minimal. The set of moveable vertices $S$ is adjustable in $|S|$ and is minimized as well. Hence, MFEC-MAR is a multi-objective optimization problem (MOP). Formally, MFEC-MAR is defined as follows:

\begin{itemize}
 \item \textbf{Input:} A sensor network $G = (V, E)$, where $V$ is a set of fixed vertices and $E$ is the set of edges between vertices if they are within transmission range $T$, a fairness measure $\alpha$, a maximum power consumption rate $P$

\item \textbf{Output:} A non-disconnected network $G' = (V \cup S, E')$ such that:
	\begin{itemize}
		\item $STDEV( PCR_{v \in |V \cup S|} v) \leq \alpha$
		\item $\forall_{v \in |V \cup S|} PCR(v) \leq P$
		\item $\sum_{v \in |V \cup S|} PCR(v)$ is minimal
		\item $|S|$ is minimal
	\end{itemize}
\end{itemize}

For the remainder of this paper, whenever we refer to fairness of power consumption rates ($PCR$), we denote this fairness measure $\alpha$ as the maximum allowable standard deviation of power consumption rates.


