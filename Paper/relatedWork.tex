\section{Related Work}\label{Related Work}

Construction and design of wireless sensor networks often heavily relies on common graph algorithms such as Prim and Kruskal's \textit{Minimum Spanning Tree (MST)}\cite{CLRS} to ensure connectivity while minimizing the total length of all edges (i.e. wireless connections). The \textit{Steiner Tree} problem extends this notion by introducing additional vertices to a graph in order to reduce the length of the spanning tree. The Steiner Tree problem is of particular interest to WSNs, in that additional vertices called relay nodes are often introduced to ensure connectivity. Estrin et al. \cite{940390} focus on connectivity establishment and maintainability using relay nodes in WSNs. They note that WSNs are subject to wireless transmission range constraints which in turn affect the interconnections between nodes. As the energy required to transmit data on wireless links is directly proportional to the distance between source and destination nodes, a significant amount of additional energy is required to establish and maintain a connection with more distant nodes. Hence, the most prominent source of energy consumption in wireless sensor networks is message transmission. Estrin et al. utilize relay node placement to combat increased energy requirements by placing nodes along wireless links of maximum distance in the network.

Lloyd \cite{1191701} discusses two strategies for ensuring a WSN is fully connected.  The first is a single-tiered node placement strategy in which every node is connected via some path consisting of either sensor nodes or relay nodes.  The length between a sensor node and any other node must be $ \leq r$, while the length between two relay nodes can be of distance $R$, where $R$ is defined as $R \geq r$.  Node placement is achieved using Steiner Tree nodes. The second strategy discussed is what they call a two-tiered node placement strategy, in which any two nodes of the WSN are fully connected by all relay nodes. That is, between any two sensor nodes in the WSN there exists a path by which every intermediary node is a relay node.

Gao et al. \cite{Gao200675} introduce the notion of reducing the radio range of nodes and employing a collaborating scheme between nodes to forward data to a base station or sink in order to conserve energy.

Song et al. \cite{Song2009} address the \textit{energy hole} problem - the problem of depleting the energy reserves of high load nodes faster than non-high load nodes - by proposing an NP hard multi-objective optimization problem (MOP). They propose a centralized algorithm as well as a distributed algorithm for assigning the transmission ranges of sensors in order to maximize the network lifetime. Notice that increasing the network lifetime also increases the lifetime the shortest lived nodes. We differ in our work in that we aim to equalized the power consumption rates of all nodes while simultaneously increasing the network lifetime.

Cheng \cite{RelaySensor} proposes a relay node placement algorithm using the minimum number of relay nodes, so that the distance between each hop is less than or equal to the common transmission range. This problem is similar to the Steiner Minimum Tree with minimum number of Steiner Points and bounded edge-length (SMT-MSPBEL). They propose a 3-approximation algorithm as well as a 2.5 approximation algorithm. The 2.5 approximation algorithm follows a randomized strategy whose performance is faster than 3-approximation algorithm.

Gandham et al. \cite{1495854} investigate energy efficiency in wireless sensor networks using multiple base stations. In order to prolong the lifetime of the sensor network, multiple base stations are employed that cover the entire network area. The lifetime of the sensor network is separated into equal periods of time called \textit{rounds} and base stations are relocated at the start of a round. They propose an algorithm for base station placement at the beginning of each round which maximizes network lifetime.

Tang et al. \cite{Tang2006490} study relay node placement problem in large scale wireless sensor networks such that sensor nodes are connected to at least one relay node and all relay nodes are connected amongst each other. To provide relay node fault tolerance they define the \textit{2-Connected Relay Node Double Cover} (2CRNDC) problem and present a polynomial time approximation algorithms to solve the 2CRNSC.

Nguyen et al. \cite{4351517} address the problem of unbalanced energy consumption among sensor nodes. This paper touches on concepts directly related to our work. They notice that unbalanced energy consumption among sensor nodes in home network domains is the result of employing only a single base station to collect sensor data. As sensor nodes further from the base station have to utilize more energy to transmit data across further distances, they propose a sensor network architecture with 2 base stations (and their corresponding communication protocol). While this solution approximately equalizes power consumption in relatively small networks, we are interested in a more general network topology.