
\documentclass[11pt,conference]{IEEEtran} 
%\documentclass[conference]{IEEEtran}
\usepackage{cite}
\usepackage[pdftex]{graphicx}
\usepackage[cmex10]{amsmath}
\usepackage{algorithmic}
\usepackage[tight,footnotesize]{subfigure}
\usepackage{url}

\begin{document}
% paper title
% can use linebreaks \\ within to get better formatting as desired
\title{Fair, Energy-aware Connectivity in Wireless Sensor Networks}
% author names and affiliations
% use a multiple column layout for up to three different
% affiliations
\author{\IEEEauthorblockN{Niels Kasch*}
\IEEEauthorblockA{Email: nkasch1@umbc.edu}
\and
\IEEEauthorblockN{Dave Feltenberger*}
\IEEEauthorblockA{Email: dfelten1@umbc.edu\\\\
*Department of Computer Science and\\Electrical Engineering\\
University of Maryland, Baltimore County\\
Baltimore, MD 21250\\
}
\and
\IEEEauthorblockN{Fatih Senel*}
\IEEEauthorblockA{Email: fsenel1@umbc.edu}
}

\maketitle



\begin{abstract}
The abstract is a substitution for the entire paper (about 250 words).

%Wireless Sensor Networks (WSN) consist of a collection of independent devices (nodes) that are connected wirelessly in an ad hoc fashion. Generally, each node is equipped with sensors collecting information from and monitoring its environment. Information gathered from WSN has had an enormous impact in the private and public sectors such as traffic monitoring and biomedication.

%Reliable connectivity in WSN has been an extensive research focus followed by energy conservation measures. As devices (nodes) in a WSN are often battery powered, battery lifetimes dictate node lifespans. The more energy can be conserved, the longer the WSN can operate without interruption.

%Previous work focuses on power optimization of the overall sensor network. The drawback of such a strategy comes from the unequal power consumption rates of different nodes. Since nodes require more energy in order to cover greater transmission distances, it follows that replacement schedules according to battery lifetimes differ and are harder to predict for individual nodes. Equalized replacement schedules are essential for biomedication application. In such a setting, it is preferable to replace all nodes in one procedure with minimal patient discomfort.

%We introduce the notion of \textit{fairness} of power consumption in WSN. We propose a two-fold optimization problem where the fairness of a measure $F_n$ for each node $n$ across a wireless sensor network is to be minimized. In addition, a WSN is to be connected in a way such that, after the addition of a minimal number of relay nodes, the spanning tree connecting the network is of minimum length and the distances between nodes are approximately equal.
\end{abstract}

\begin{keywords}
Wireless Sensor Network, Energy Conservation, Fairness, Energy Consumption
\end{keywords}




\IEEEpeerreviewmaketitle


\section{Introduction}

Over the past decade, research in the area of Wireless Sensor Networks (WSN) has produced valuable insight in the private, public and medical sectors. WSNs consist of a collection of independent devices (nodes) that are connected wirelessly in an ad hoc fashion. Generally, each node is equipped with sensors collecting information from and monitoring its environment. Evaluation of such information enables reactive, proactive and modeling capabilities each of which is utilized in fields such as traffic control, home automation and biomedication, to name a few.

The size of WSNs ranges from a few nodes to upwards of 10000 nodes covering areas beyond wireless transmission capabilities. Multi-hop networking is an integral part in these networks to ensure connectivity between all nodes. Connectivity is by far the most important criterion involved in WSN design decisions. Extensive research has been conducted to:
\begin{itemize}
\item identify ways to interconnect all nodes with a minimum number of links. The Minimum Spanning Tree (MST) adopted from graph theory and general networking addresses this issue.
\item interconnect nodes by a network of shortest length when additional nodes (i.e. relay nodes) are introduced. This problem has been studied as the Steiner Tree problem.
\item optimize a network of shortest length according to a restricting parameter. In WSN, the transmission range of a node may impose limits on the length of a link.
\end{itemize}

A second important criterion, which is the focus of our work, is the energy efficiency of the network. Nodes in a WSN are typically battery powered and as such their lifetime can be measured by their average battery lifetime. In medical applications such as patient embedded sensors, battery conservation becomes a matter involving the quality of life for a patient. The shorter the lifetime of a sensor node, the more often a patient must undergo node replacement procedures. 

Wireless radio transmissions drop off at an exponential rate due to ground reflection of radio signals. The further a signal should travel to more energy must be utilized. Since the signal attenuation rate is exponential with distance, a significant amount of energy must be used to transmit a signal only a small distance further. Chang et. al \cite{RelaySensor} investigated WSN power consumption optimization by imposing a global maximum transmission range and employing Steiner Points in the resulting Steiner Minimum Tree as relay nodes to ensure connectivity. Using this technique, \cite{RelaySensor} were able to conserve a significant amount of energy in the overall WSN topology.

While overall power consumption is certainly an improvement over previous architectures, we are interested in identifying approximately equal power consumption for every node in the network. Our work would have an important impact on WSN where predictability of equal power consumptions is important. For example, if the battery lifetime of every node in a WSN for patient biomedication is approximately equal and maximized, then a patient will undergo the fewest number of  node replacement procedures. During such a procedure, every node can be replaced as battery lifetimes are approximately equal. This idea differs from previous work in that power consumption of the network as a whole is equalized and optimized as opposed to only optimizing the power consumption which may leave individual nodes with differing power consumption rates. Hence, we introduce the notion of \textit{fairness} of power consumption in WSN. We, therefore, propose a two-fold optimization problem as follows:
Optimize the fairness of a measure $F_n$ for each node $n$ across a wireless sensor network, given that $F_n$ is exponentially related to a distance measure between neighboring nodes. Thus, $F_n$ is to be minimized. Furthermore, the network is to be connected in a way such that, after the addition of a minimal number of relay nodes, the spanning tree connecting the network is of minimum length and the distances between nodes are approximately equal.

\subsection{Problem Formulation}
Specifies the problem (i.e. fairness in WSN) we are addressing in this paper. Defines the key concepts, such a fairness in relation to power consumption and efficiency of power consumption as well as addresses pitfalls such as minimal power consumption which is not addressed by this paper.
\subsection{System Model}
The system model is explained. All assumptions required for the model are specified.

Important performance metrics, such as:
\begin{itemize}
\item \texttt{time to death of the first node}(TTFD)
\item \texttt{average node lifetime} (ANL)
\item \texttt{cut vertex lifetime} and its optimality
\item additional resource requirements such as \texttt{number of Steiner points} and additional nodes
\item $L$ - maximum power consumption rate
\item $K$ - number of new Steiner Points
\item $\alpha$ - minimum standard deviation of power consumption rates given $K$ Steiner points
\end{itemize}
are defined in this section.
\section{Related Work}

Cheng \cite{RelaySensor} proposed a relay node placement algorithm using the minimum number of relay nodes, so that the distance between each hop is less than or equal to the common transmission range. This problem is similar to the Steiner Minimum Tree with minimum number of Steiner Points and bounded edge-length (SMT-MSPBEL). They proposed a 3-approximation algorithm and a 2.5 approximation algorithm. The 2.5 approximation algorithm is a randomized algorithm whose performance is faster than 3-approximation algorithm.

Lloyd \cite{1191701} discusses two strategies for ensuring a WSN is fully connected.  The first is a single-tiered node placement strategy in which every node is connected via some path consisting of either sensor nodes or relay nodes.  The length between a sensor node and any other node must be <= r, while the length between two relay nodes can be of distance R, where R is defined as R >= r.  Node placement is achieved using Steiner Tree nodes.

The second strategy discussed is what they call a two-tiered node placement strategy, in which any two nodes of the WSN are fully connected by all relay nodes. That is, between any two sensor nodes in the WSN there exists a path by which every intermediary node is a relay node.

Recent research on relay node placement in Wireless Sensor Networks mostly focuses on connectivity establishment (Estrin et al. \cite{940390}). Connectivity is one of the most crucial issue, since most of the actions are taken collaboratively. In addition, sensors are small battery operated devices. Usually after the deployment of wireless sensor networks, it is impossible to access a sensor node and change the battery. Once the battery is depleted the node dies. The most important source of energy consumption in sensor networks is message transmission. The energy required to transmit data is directly proportional to $r^4$ where $r$ is the distance between source and destination.

NEED TO REFER TO THIS PAPER\cite{596303}.
\section{Proposed Approach}\label{approach}

\subsection{System Model}
The model and algorithms described in this paper make the following assumptions:

\begin{itemize}
\item Each sensor/node sends an equal amount of data per time unit.
\item Each node can receive an arbitrary amount of data per time unit.
\item Each node can store an arbitrary amount of data without energy penalties. This assumption enables this model to ignore data transmission bottlenecks.
\item The transmission range of each sensor node does not exceed the maximum transmission range $T$.
\item Only nodes within transmission range of each other are connected (i.e. have an edge between each other in the network).
\item A node's energy dissipation per bit transmitted (according to the first order radio model \cite{820485}).
\item Transmitting a bit over distance $d$ requires energy $d^2$.
\item Sensor nodes have a fixed location. Their location may not change during the lifetime of the network.
\item Relay nodes are movable. Their location is flexible during the lifetime of the network.
\item Relay nodes can be added to the original network.
\end{itemize}

\subsection{Problem Formulation}

In the Minimal Fair Energy Consumption with Minimal Additional Resources (MFEC-MAR) problem the goal, given a wireless sensor network represented as a graph $G=(V \cup S,E)$, is to equalize the power consumption rate $PCR$ of all nodes in $G$. That is $PCR(v_1)=PCR(v_2)=PCR(v_3)=...=PCR(v_n)$ where $n=|V \cup S|$ denotes the number of vertices/nodes in $G$. The set of vertices in $G$ is composed of a set of fixed vertices $V$ (i.e. sensor nodes) and a set of moveable vertices $S$ (i.e. relay nodes). The edges $E$ in $G$ are subject to the maximum transmission range $T$ of the nodes. It is furthermore the goal of MFEC-MAR to minimize overall power consumption of the network such that $\sum_{i=1}^{n} PCR(v_i)$ is minimal. The set of moveable vertices $S$ is adjustable in $|S|$ and is minimized as well. Hence, MFEC-MAR is a multi-objective optimization problem (MOP). Formally, MFEC-MAR is defined as follows:

\begin{itemize}
 \item \textbf{Input:} A sensor network $G = (V, E)$, where $V$ is a set of fixed vertices and $E$ is the set of edges between vertices if they are within transmission range $T$, a fairness measure $\alpha$, a maximum power consumption rate $P$

\item \textbf{Output:} A non-disconnected network $G' = (V \cup S, E')$ such that:
	\begin{itemize}
		\item $STDEV( PCR_{v \in |V \cup S|} v) \leq \alpha$
		\item $\forall_{v \in |V \cup S|} PCR(v) \leq P$
		\item $\sum_{v \in |V \cup S|} PCR(v)$ is minimal
		\item $|S|$ is minimal
	\end{itemize}
\end{itemize}

For the remainder of this paper, whenever we refer to fairness of power consumption rates ($PCR$), we denote this fairness measure $\alpha$ as the maximum allowable standard deviation of power consumption rates.


\subsection{Algorithms for MFEC-MAR (Fairness in WSNs)}
In this section we will explain our two step approach to solving the fairness problem in WSNs.

\subsubsection{Connecting Nodes}
This section explains how to connect and initially disconnected set of nodes using the Steiner Minimum Tree (SMT) algorithm.
\subsubsection{Optimizing Node Location}
In this section we explain an extension to the SMT problem to include a fairness measure across the network of nodes.
The explanation will include the moving of Steiner Points to optimal locations as well as the introduction of additional resources (nodes) to achieve optimality. The discussion includes an overview of the trade-off between additional resources and  fairness and illustrates results of the optimization operations across multiple dimensions (such as minimizing the deployment of additional resources, maximizing fairness while minimizing power consumption).
\section{Analysis}\label{analysis}

The section compares our approach the 3-approximation algorithm for minimal power consumption of WSNs.
\section{Theoretical Analysis}

Our approach will be compared to the 3-approximations time and space usages from a theoretical standpoint.

\section{Experimental Analysis}
The theoretical analysis will be verified by an experimental analysis.


\section{Conclusion}\label{conclusion}
Concluding remarks summarizing our approach and future work will be outlined in this section.

\section*{Acknowledgment}

The authors would like to thank...

\bibliographystyle{IEEEtran}
\bibliography{IEEEfull,paper}



% that's all folks
\end{document}



