\section{Algorithm and Analysis}\label{AlgorithmAndAnalysis}

\subsection{Algorithms for MFEC-MAR (Fairness in WSNs)}
In this section we will explain our two step approach to solving the fairness problem in WSNs.

\subsubsection{Connecting Nodes}\label{connectingNodes}
Given a set $T$ of n terminals which are deployed in an Euclidean plane, and a positive constant R, our aim is to find a Steiner Minimum Tree $\tau$ with minimum number of Steiner points (SMT-MSP). In 1999, Lin \textit{et al}. \cite{308672} showed that SMT-MSP is NP-Hard. In 2008, Cheng \textit{et al}. \cite{RelaySensor}, presented a $O(n^{3})$-time approximations with performance ratio is at most 3. In this project we implemented Ratio-3 approximation presented in \cite{RelaySensor}, to connect initially disconnected terminals. The idea behind the algorithm is simple: First we assume a fully connected undirected graph $G=(V,E)$ where $V=$ set of terminals, and $(u,v)\in E$ $\forall u,v \in V$ and $u\neq v$. Then we sort all $\frac{n\cdot(n-1)}{2}$ edges. For each subset of 3 terminals a, b and c respectively in three connected components, if there exists such a point $s$ within distance R, we put a 3-star $s$ which includes edges $(s,a)$, $(s,b)$, and $(s,c)$. Then for each edge where $\left|e_i\right| > R$ and $e_i$ connects two connected component, we move the steinerized $e_i$ into $\tau$. Thus we establish the connectivity of terminals.

\linesnumbered
\begin{algorithm}[htp]

\dontprintsemicolon
\SetKwData{FunctionSignature}{FairSMT}
\FunctionSignature{G, $\alpha$, k, R}
\\
\KwIn{$G=(T,E)$ such that $T$ is the set of terminals and $E$ is the initial set of edges, $\alpha$ is the target Standard Deviation, k is the maximum number of relay nodes, and R is the positive constant which indicates the radio range of a sensor}
\KwOut{$G'=(V,E')$ with $V=T \cup S$ where $S$ is the set of relay nodes and $E'$ is the new set of edges}
\Begin{
$G' \leftarrow Ratio-3-SMT(G)$

$G' \leftarrow MakeFair(G', \alpha, k)$

return $G'$\\
}
\caption{FairSMT}
\label{FairSMT}
\end{algorithm}


\subsubsection{Optimizing Node Location}\label{OptimizingNodeLocation}
SMT-MSP provides connectivity using minimum number of Steiner points. In some cases, the relay nodes in the resulting topology, may be too close or too far from each other. However this is not desirable, since it violates fairness of power consumption rates.  \\

In this section we explain an extension to the SMT problem to include a fairness measure across the network of nodes.
The explanation will include the moving of Steiner Points to optimal locations as well as the introduction of additional resources (nodes) to achieve optimality. The discussion includes an overview of the trade-off between additional resources and  fairness and illustrates results of the optimization operations across multiple dimensions (such as minimizing the deployment of additional resources, maximizing fairness while minimizing power consumption).\\ 


%Algorithm 2
\begin{algorithm}[ht]

\dontprintsemicolon
\SetKwData{FunctionSignature}{MakeFair}
\FunctionSignature{$G'$, $\alpha$, k}
\\
\KwIn{$G=(V,E)$ with $V=T \cup S$ where $T$ is the set of terminals and $S$ is set of nonterminals returned by Ratio-3 approximation, $\alpha$ is the target Standard Deviation, k is the maximum number of relay nodes}
\KwOut{$(G'=(V',E'), achieved)$ where $V'=V \cup S'$ where $S'$ is the set of newly added relay nodes and $E'$ is the new set of edges, and achieved is boolean variable that indicates if target Standart Deviation is achieved or not}
\Begin{
  // build a max-heap with the PCR of each steiner node\\
  $heap \leftarrow Max-Heap(S)$\\
  \While {$\left|S\right| < k$}{
     $\beta \leftarrow $STDEVofPCRs()\\
     $dec \leftarrow true$\\
     \While {$dec=true$}{
        $MoveRelayNode-geometric(G)$\\
        \eIf {STDEVofPCRs() $< \beta$}{
           $dec \leftarrow true$\\
           $\beta \leftarrow$ STDEVofPCRs()
        }
        {
           $dec \leftarrow false$
        }
     }
     $AddRelayNode(G)$\\
     $MoveRelayNodes(G)$\\
  }
  $G' \leftarrow G$\\
  $\beta \leftarrow$ STDEVofPCRs()\\
  \eIf {$\beta \leq \alpha$}{
     return $(G', true)$
  }{
  return $(G', false)$
  }
}
\caption{Pseudo-code of fairness approximation}
\label{MakeFair}
\end{algorithm}


%Algorithm 3
\begin{algorithm}[htp]

\dontprintsemicolon
\SetKwData{FunctionSignature}{MoveRelayNodes-geometric}
\FunctionSignature{$G$}
\\
\KwIn{$G=(V,E)$ with $V=T \cup S$ where $T$ is the set of terminals and $S$ is set of nonterminals}
\KwOut{$G=(V,E)$ same graph with different edge lengths}
\Begin{
  \For{$s_i \in S$}{
     \If{$s_i$ has 3 neighbors}{
        \If{neighbors of $s_i$ is on a circle }{
           move $s_i$ to the center of the circle\\
        }
     }
  }
}
\caption{Detailed description of MoveNodes-geometric function}
\label{MoveRelayNodes-geometric}
\end{algorithm}


%Algorithm 4
\begin{algorithm}[htp]

\dontprintsemicolon
\SetKwData{FunctionSignature}{AddRelayNodes}
\FunctionSignature{$G$}
\\
\KwIn{$G=(V,E)$ with $V=T \cup S$ where $T$ is the set of terminals and $S$ is set of nonterminals}
\KwOut{$G'=(V',E')$ where $V'=V \cup u$ and $u$ is the newly added relay node and $E'$ is the new set of edges, and achieved is boolean variable that indicates if target Standart Deviation is achieved or not}
\Begin{
	$s \leftarrow$ heap.extractMax()\\
	$t \leftarrow$ s.getFarthestNeighbor()\\
	remove $(s,t)$ from the steiner tree.\\
	deploy new steiner node $u$\\
	$p  \leftarrow$ middle point $(s, t)$\\
	$u$.setCoordinates$(p)$\\
	add $(s,u)$ and $(u,t)$ to the tree\\
	heap.insert$(u)$\\
	heap.insert$(s)$\\
	heapify()\\
}
\caption{Pseudo-code of AddSteinerNodes}
\label{AddRelayNodes}
\end{algorithm}

%Algorithm 5
\begin{algorithm}[htp]

\dontprintsemicolon
\SetKwData{FunctionSignature}{MoveRelayNodes}
\FunctionSignature{$G$}
\\
\KwIn{$G=(V,E)$ with $V=T \cup S$ where $T$ is the set of terminals and $S$ is set of nonterminals}
\KwOut{$G=(V,E)$ same graph with different edge lengths}
\Begin{
	\While {heap is not empty}{
		$s \leftarrow$ heap.extractMax()\\
		$t \leftarrow s$.getFarthestNeighbor()\\
		// calculateLocation finds the best location $p$ which is in between $s$ and $t$, where $p$ minimizes the PCR of $s$.\\
		$p \leftarrow $calculateLocation$(s,t)$\\
		move$(s, p)$\\
		heap.insert$(s)$\\
		heapify()\\
	}
}
\caption{Pseudo-code of MoveRelayNodes}
\label{MoveRelayNodes}
\end{algorithm}


Initially, our fairness approximation algorithm calculates the power consumption rates of each relay node (PCR) and inserts the nodes to a max heap. The main loop of the MakeFair method iteratively moves and adds relay nodes until the max number relay nodes allowed, is reached. The inner while loop of Algorithm 2 (lines 8-16), invokes MoveRelayNodes-geometric function, which is described in Algorithm 3. MoveRelayNodes-geometric takes each 3-stars as defined in \cite{RelaySensor}, and moves the relay nodes to the center of circle that circumscribes the 3-neighbors. Here, our aim is to provide local fairness. After moving each relay node to the center of 3-star, we extract the node which has the highest PCR, and find its farthest neighbor (see Algorithm 4). Since the longest edge determines the PCR of the node, we add an additional relay node to the middle point the longest edge. Then we add newly deployed node along with the extracted node to the heap. Deploying new node will definitely change PCRs of some set of nodes. We move each relay node to the local optimal positions. In Algorithm 5, it is seen that, the optimal location for a node $u$ is in between itself and its farthest neighbor. We calculate this location using some trigonometric functions. Then we insert $u$ to the heap and continue iteration until $\left|S\right|=k$ where $S$ is the set of relay nodes and $k$ is the maximum number of relay nodes which can be deployed.


\subsection{Theoretical Analysis}

In Fairness Approximation we first create a heap and insert the relay nodes to the heap. Assuming that FIB-HEAP structure is used to implement the MAX-HEAP. Since amortized cost single insert operation is $O(1)$, the cost of inserting $k$ elements into the heap is $O(k)$ where k is maximum number of relay nodes which can be deployed. In the worst case, the outer while loop of Algorithm iterates $k$ times and inner while loop iterates $\beta$ times where $\beta$ is the initial standard deviation of PCRs. The time complexity of MoveRelayNodes-geometric is $O(k)$. So the total cost of inner while loop (lines 8-16 of Algorithm 2) is $O(k\beta)$ \\

In Algorithm 4, we extract the node with highest PCR from the heap, and add a relay node to middle of the longest edge. The cost of extracting from the heap is $O(\lg k)$. In Algorithm 5, we move each and every relay node to their local optimal locations. The time complexity of MoveRelayNodes function is $O(k \lg k)$.\\

The overall time complexity is $O(n^{3})+O(k)(O(k\beta)+O(\lg k)+O(k \lg k))=O(n^3)+O(k^{2}\beta)+O(k^{2} \lg k)$

\subsection{Experimental Analysis}
The theoretical analysis will be verified by an experimental analysis.


