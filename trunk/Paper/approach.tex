\section{Proposed Approach}\label{approach}

\subsection{Problem Formulation}
Specifies the problem (i.e. fairness in WSN) we are addressing in this paper. Defines the key concepts, such a fairness in relation to power consumption and efficiency of power consumption as well as addresses pitfalls such as minimal power consumption which is not addressed by this paper.
\subsection{System Model}
The system model is explained. All assumptions required for the model are specified.

Important performance metrics, such as:
\begin{itemize}
\item \texttt{time to death of the first node}(TTFD)
\item \texttt{average node lifetime} (ANL)
\item \texttt{cut vertex lifetime} and its optimality
\item additional resource requirements such as \texttt{number of Steiner points} and additional nodes
\item $L$ - maximum power consumption rate
\item $K$ - number of new Steiner Points
\item $\alpha$ - minimum standard deviation of power consumption rates given $K$ Steiner points
\end{itemize}
are defined in this section.


\subsection{Algorithms for Fairness in WSNs}
In this section we will explain our two step approach to solving the fairness problem in WSNs.
\subsubsection{Connecting Nodes}
This section explains how to connect and initially disconnected set of nodes using the Steiner Minimum Tree (SMT) algorithm.
\subsubsection{Optimizing Node Location}
In this section we explain an extension to the SMT problem to include a fairness measure across the network of nodes.
The explanation will include the moving of Steiner Points to optimal locations as well as the introduction of additional resources (nodes) to achieve optimality. The discussion includes an overview of the trade-off between additional resources and  fairness and illustrates results of the optimization operations across multiple dimensions (such as minimizing the deployment of additional resources, maximizing fairness while minimizing power consumption).