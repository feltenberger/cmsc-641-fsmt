\section{Proposed Approach}\label{approach}

\subsection{System Model}
The model and algorithms described in this paper make the following assumptions:

\begin{itemize}
\item Each sensor/node sends an equal amount of data per time unit.
\item Each node can receive an arbitrary amount of data per time unit.
\item Each node can store an arbitrary amount of data without energy penalties. This assumption enables this model to ignore data transmission bottlenecks.
\item The transmission range of each sensor node does not exceed the maximum transmission range $T$.
\item Only nodes within transmission range of each other are connected (i.e. have an edge between each other in the network).
\item A node's energy dissipation per bit transmitted (according to the first order radio model \cite{820485}).
\item Transmitting a bit over distance $d$ requires energy $d^2$.
\item The power consumption rate $PCR$ for a node $n$ is defined as $PCR(n) = \frac{\max_{v \in N}d(v)^2}{\textrm{unit time}}$, where $N$ is the set of neighbors of a node.
\item Sensor nodes have a fixed location. Their location may not change during the lifetime of the network.
\item Relay nodes are movable. Their location is flexible during the lifetime of the network.
\item Relay nodes can be added to the original network.
\end{itemize}

\subsection{Problem Formulation}

We formulate the Minimal Fair Energy Consumption with Minimal Additional Resources (MFEC-MAR) problem as follows: Given a set of fixed sensor nodes $S$, find a minimal set of relay sensor nodes $R$ such that the power consumption rate $PCR$ is equal for each node $v \in \{S \cup R\}$, that is:
\begin{equation*}
\forall_{i=0}^{n=|S \cup R|} PCR(v_i) = PCR(v_2)=...=PCR(v_n).
\end{equation*}
The induced graph $G=(S \cup R,E)$ must be connected, where an edge $(u,v) \in E$ if $u$ is within transmission range $T_v$ of $v$ and $v$ is within transmission range $T_u$ of $u$. A node $u$ is within transmission range of $v$ if  $PCR(u)\leq distance(u,v)^2$.
Formally, MFEC-MAR is defined as follows:
\begin{itemize}
 \item \textbf{Input:} A set of fixed sensor nodes $S$ in Euclidean space.
\item \textbf{Output:} A connected network $G = (S \cup R, E)$ such that:
	\begin{itemize}
		\item $R$ is the set of introduced relay nodes such that:
		\begin{itemize}
		\item $|R|$ is minimal
		\item $\forall_{i=0}^{n=|S \cup R|} PCR(v_i) = PCR(v_2)=...=PCR(v_n)$
		\end{itemize}
		\item $G$ is connected
		\item  $\sum_{v \in |S \cup R|} PCR(v)$ is minimal
	\end{itemize}
\end{itemize}

The Minimal Fair Energy Consumption with Minimal Additional Resources Approximation (MFEC-MAR-Approx) problem aims to find an approximate solution to the MFEC-MAR problem. In the latter part of this paper, we will give an algorithm for MFEC-MAR-Approx. The MFEC-MAR-Approx problem is defined as follows:

Given a set of fixed sensor nodes $S$, an allowable standard deviation $\alpha$ and a maximum number of relay nodes $k$, find a set of relay sensor nodes $R$, where $|R|\leq k$, such that the power consumption rates $PCR$ for all nodes is within $\alpha$. The induced graph $G=(S \cup R,E)$ must be connected, where an edge $(u,v) \in E$ if $u$ is within transmission range $T_v$ of $v$ and $v$ is within transmission range $T_u$ of $u$. A node $u$ is within transmission range of $v$ if  $PCR(u)\leq distance(u,v)^2$.
Formally, MFEC-MAR-Approx is defined as follows:

\begin{itemize}
 \item \textbf{Input:} A set of fixed sensor nodes $S$ in Euclidean space, a standard deviation $\alpha$ of power consumption rates and a maximum number of relay nodes $k$.
\item \textbf{Output:} A connected network $G = (S \cup R, E)$ such that:
	\begin{itemize}
		\item $R$ is the set of introduced relay nodes such that:
		\begin{itemize}
		\item $|R| \leq k$
		\item $\sigma(PCR) \leq \alpha$		
		\end{itemize}
		\item $G$ is connected
	\end{itemize}
\end{itemize}

For the remainder of this paper, whenever we refer to fairness of power consumption rates ($PCR$), we denote this fairness measure $\alpha$ as the maximum allowable standard deviation of power consumption rates.


